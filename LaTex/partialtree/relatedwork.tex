\section{Related Work}

Many papers have addressed the topic of implementations of XPath queries 
in parallel. One significant paper was presented by IBM~\cite{BLKK09}. The paper 
proposes three kinds of strategies for XPath queries in 
parallel: data partition strategy, query partition strategy, 
and hybrid partition strategy. Many papers can be categorized to one of the strategies. \cite{KrYa10,ZhPC10,HSYW14} focus on XPath queries 
implemented in a shared-memory environment. \cite{PLZC07,WZYL08}
focused on XML parsing, which is related to our parsing 
algorithm. \cite{AnAH10} proposed ideas about XML processing 
techniques that are helpful for our research. 
Some prior researches are based on a common assumption 
that a large amount of XPath queries are executed over an XML stream.
YFilter\cite{DiFF11} and XMLTK \cite{AvGT02} execute thousands of small queries in parallel. The parsing phase is still sequential.
Indexing is also a hot topic for improving the performance of parallel XML queries processing.
\cite{CVZT02}, \cite{Grus02}, \cite{JLCW02} are related to this field. 
They examined the indices on different types of trees, including B+-tree, R-tree, and XR-tree. 

The idea of dividing the XML documents and running the computation for trees
with the chunks is not new.   Kakehi et al.~\cite{KaME07} showed a parallel tree reduction algorithm
from the nodes in chunks.  Based on the idea given by Kakehi et al., Emoto and Imachi~\cite{EmIm12} developed 
a parallel tree reduction algorithm on Hadoop, and Matsuzaki and Miyazaki~\cite{MaMi15} developed 
a parallel tree accumulation algorithm. A similar approach was taken by Sevilgen et al.~\cite{SeAF05} who developed a simpler version of tree accumulations over the serialized representation of trees.

It is known that we can develop a parallel algorithm for XPath queries using the tree accumulations~\cite{Mats07}.
The approach we took in this paper is inspired by the work by Morihata~\cite{Mori13}.
To discuss the advantages of the proposed algorithm and compare by implementation with other approaches
are our important future work.
