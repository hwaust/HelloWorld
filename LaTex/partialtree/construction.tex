%\def\INDEXSET#1{\mathit{#1}_{[0,P)}}
\def\INDEXSET#1{\mathit{#1}_{[P]}}
\def\baselinestretch{1.5}

\section{Partial Tree Construction}
\label{sec:construction}

Since the structure of a partial tree is different from ordinary XML trees, 
we designed an algorithm for partial tree construction.
The algorithm for constructing partial trees has three steps: constructing subtrees from parsing chunks, pre-path computation, and computation for ranges. 

\subsection{Construction of subtrees from parsing XML chunks}

A partial tree is constructed from parsing an input XML chunk, which is  
a substring created from splitting an XML document. We design an algorithm that parses 
the input XML string into a similar tree by using an iterative function with 
a stack. We use an example XML document listed below to demonstrate how our algorithm works.

\begin{quote}\tt\small
<A><B><C><E></E></C><D></D></B><E></E><B><B><D><E>\\
</E></D><C></C></B><C><E></E></C><D><E></E></D></B>\\
<E><D></D></E><B><D></D><C></C></B><B></B></A>\\
\end{quote}
 
From the document, we can create an XML tree as shown in Fig.~\ref{fig:tree}.
We number these nodes in a prefix order for identification.

\begin{figure*}[t]
	\centering\includegraphics[]{partialtree/figures/fromWord-5.pdf}
	\caption{an XML tree from the given XML string}
	\label{fig:tree}
\end{figure*}

Then, we split the document into five chunks as listed below.

\begin{itemize}
	\item[chunk$_0$:] \texttt{\small <A><B><C><E></E></C><D></D></B>}
	\item[chunk$_1$:] \texttt{\small <E></E><B><B><D><E></E></D>}
	\item[chunk$_2$:] \texttt{\small <C></C></B><C><E></E></C><D>}
	\item[chunk$_3$:] \texttt{\small <E></E></D></B><E><D></D></E>}
	\item[chunk$_4$:] \texttt{\small <B><D></D><C></C></B><B></B></A>}
\end{itemize}

When splitting an XML document, we need to deal with nodes with missing tags.
During parsing, we push the start tag onto the stack. When we meet an
end tag, we pop the last tag to merge a closed node. However, as a result of splitting, 
some nodes miss their matching tags. In this case, we  
mark it left-open or right-open based on which part is missing. Then, we add them onto 
the subtrees in the same way as we add closed nodes.

We also need to handle the case when the split position falls inside a tag 
and thus splits the tag into two halves.
In this case, we simply merge the split tags. 
Because there are at most two split tags on a partial tree, 
the time taken for merging them is negligible. 

One or more subtrees can be constructed from one chunk.
We construct nine subtrees by parsing the five chunks above as shown in  Fig.~\ref{fig:partialtree-notfinished}.
Chunk$_0$ and chunk$_4$ have only one subtree while chunk$_2$ has three subtrees.
After the parsing phase, these subtrees are used for pre-path computation.

\begin{figure*}[t]
	\centering\includegraphics[]{partialtree/figures/fromWord-6.pdf}
	\caption{Subtrees from parsing chunks}
	\label{fig:partialtree-notfinished}
\end{figure*}



\subsection{Pre-path computation}

The basic idea of computing the pre-paths for each partial tree is to 
make use of open nodes, because missing parent and ancestor nodes 
is caused by splitting these nodes. Therefore, the information needed 
for creating the pre-paths lies in these open nodes. 

Algorithm 0 outlined the pseudo codes for pre-path computation. 
Because one chunk may generate more than one subtrees, the input is a list of subtree lists. 
The length of the list is equal to the number of partial trees, thus represent the length as $P$.

Algorithm 0 has three phases.
The first phase selects all left-open nodes to $LLS$ and all right-open nodes to $RLS$ (line 2-4). 
$LLS_{[P]}$ collect the left open nodes of the $p$th partial tree, likewise we have $RLS_{[P]}$.
Note that the nodes in $LLS_{[P]}$ or $RLS_{[P]}$ are arranged in order from root to leaves.
For example, in Table~\ref{table:opennodes}, we select all the open nodes and add them 
to corresponding lists. 
 
 \begin{figure}[t]
 	\centering
 	\label{fig:ppalgorithm}
 	\begin{tabular}{l}
 		\hline
 		\hline
 		\makebox[.95\linewidth][l]{\textbf{Algorithm 0} \textsc{GetPrepath}($\mathit{STS}$)} \\
 		\hline
 		\textbf{Input}: $\mathit{STS}$: a list of subtree lists \\
 		\textbf{Output}: an indexed set of partial trees\\
 		\makebox[1em][r]{1:}\hspace{1 mm} /* open nodes in LLS or RLS are arranged in top-bottom order */\\
 		\makebox[1em][r]{2:}\hspace{1 mm} \textbf{for all} $p \in [0, P)$ \textbf{do} \\
 		\makebox[1em][r]{3:}\hspace{4 mm} $\INDEXSETL{LLS} \leftarrow \mathit{SelectLeftOpenNodes}(\INDEXSETL{STS})$\\
 		\makebox[1em][r]{4:}\hspace{4 mm} $\INDEXSETL{RLS} \leftarrow \mathit{SelectRightOpenNodes}(\INDEXSETL{STS})$\\
 		
 		\makebox[1em][r]{5:}\hspace{1 mm} /* Prepath-computation and collecting matching nodes */\\
 		\makebox[1em][r]{6:}\hspace{1 mm} $AuxList \leftarrow []$ \\
 		\makebox[1em][r]{7:}\hspace{1 mm} \textbf{for} $ p \in [0, P-1) $ \textbf{do}\\
 		\makebox[1em][r]{8:}\hspace{4 mm} $AuxList.\mathit{AppendToHead}(\INDEXSETL{RLS})$\\
 		\makebox[1em][r]{9:}\hspace{4 mm} $AuxList.\mathit{RemoveLast}(LLS_{[p+1]}.\mathit{Size}()) $\\
 		\makebox[1em][r]{10:}\hspace{4 mm} $ PPS_{[p+1]}\leftarrow AuxList$ \\
 		
 		\makebox[1em][r]{11:}\hspace{1 mm}  /* Add pre-nodes to subtrees */ \\
 		\makebox[1em][r]{12:}\hspace{1 mm} $PTS \leftarrow []$ \\
 		\makebox[1em][r]{13:}\hspace{1 mm} \textbf{for} $ p \in [0, P) $ \textbf{do}\\
 		\makebox[1em][r]{14:}\hspace{4 mm} \textbf{for} $ i \in [0, PPS_{[p]}.Size() - 1)$ \textbf{do} \\
 		\makebox[1em][r]{15:}\hspace{8 mm}   $PPS_{[p][i]}.children.$Add$(PPS_{[p][i+1]})$ \\
  		\makebox[1em][r]{16:}\hspace{4 mm} $PPS_{[p]}.last.children.$Add$(\INDEXSETL{STS})$ \\
 		\makebox[1em][r]{17:}\hspace{4 mm}   $\INDEXSETL{PTS} \leftarrow PPS_{[p][0]}$ \\
 		\makebox[1em][r]{18:}\hspace{1 mm} \textbf{return} \emph{PTS} \\
 		\hline
 	\end{tabular}
 \end{figure}

In the second phase, we do the pre-path computation. Once we split an XML document from a position
inside the document, the two partial tree created from splitting have the same number of open nodes 
on the splitting side. Given two consecutive partial trees, the number of right-open nodes of the left partial tree is the same as the number of left-open nodes of the right partial tree. 
We can use this feature to compute pre-paths for partial trees.
We first add the $p$th $RLS$ to the head of an auxiliary list $AuxList$ (line 8), 
then remove the same number of nodes as the number of $(p-1)$th $LLS$ (line 9). 
Last, we keep the nodes in the $AuxList$ to the $(p+1)$th $PPS$, which holds the pre-nodes for each partial tree.
Table~\ref{table:tempresult} shows the results of pre-path computation for the given example.

\begin{table}[t]
	\caption{Open node lists}
	\label{table:opennodes}
	\centering
	\begin{tabular}{c|cc}
		\hline
		& Left-open nodes	& Right-open nodes \\
		\hline
		pt$_0$	& []			& [A$_0$] \\
		pt$_1$	& []			& [B$_6$, B$_7$] \\
		pt$_2$	& [B$_7$]			& [D$_1$$_3$] \\
		pt$_3$	& [B$_6$, D$_1$$_3$]		& [] \\
		pt$_4$	& [A$_0$]			& [] \\
		\hline
	\end{tabular}
\end{table}
 
\begin{table}[t]
	\caption{Results of pre-path computation in AUX}
	\label{table:tempresult}
	\centering
	\begin{tabular}{c|ccc}
		\hline
		& Left-open nodes	& Right-open nodes & $AUX$\\
		\hline
		pt$_0$	& []			& [A$_0$]  & []\\  
		pt$_1$	& []			& [B$_6$, B$_7$] & [A$_0$] \\
		pt$_2$	& [B$_7$]			& [D13] &[A$_0$,B$_6$]\\
		pt$_3$	& [B$_7$, D$_1$$_3$]		& []  & [A$_0$]\\
		pt$_4$	& [A$_0$]			& [] &[]\\
		\hline
	\end{tabular}
\end{table}


In last phase, we add pre-nodes to the corresponding partial tree and copy the nodes in $PPS_{[p]}$ to $PTS_{[p]}$ as results for output.Because pre-nodes in the pre-path are also open nodes, we list all open nodes for each partial trees in Table~\ref{table:allopennodes}. Then, the pre-path computation is complete. For the given example, we obtain the partial trees as shown in Fig.~\ref{fig:partialtree2}.

\begin{table}[t]
	\caption{All open nodes}
	\label{table:allopennodes}
	\centering
	\begin{tabular}{c|cc}
	\hline
	& Left-open nodes	& Right-open nodes \\
	\hline
	pt$_0$	& []						& [A$_0$] \\
	pt$_1$	& [A$_0$]					& [A$_0$, B$_6$, B$_7$] \\
	pt$_2$	& [A$_0$, B$_6$, B$_7$]		& [A$_0$, B$_6$, D$_1$$_3$] \\
	pt$_3$	& [A$_0$, B$_6$, D$_1$$_3$]	& [A$_0$] \\
	pt$_4$	& [A$_0$]					& [] \\
	\hline
	\end{tabular}
\end{table} 


\begin{figure*}[t]
	\centering\includegraphics[]{partialtree/figures/fromWord-7.pdf}
	\caption{Partial trees from the given XML string.}
	\label{fig:partialtree2}
\end{figure*}

\subsection{Creation of Ranges of Open Nodes}  
Once an XML node is split, it generates two or more open nodes
on consecutive partial trees. For example, as we can see in Fig.~\ref{fig:partialtree2}, 
\Nr{B}{6} on pt$_1$, \Np{B}{6} on pt$_2$, and \Nl{B}{6}  on pt$_3$ 
are created from the same node \Nc{B}{6}.
For locating the open nodes of the same node on different partial trees, 
we use two integers \textit{start} and \textit{end} for the open nodes. With these two integers, 
we can decide the partial trees that have matching nodes of the same open node.
Note that after adding nodes to a partial tree, the nodes from the same node also have the 
same depth. Therefore, we can locate all the matching nodes to set \textit{start} and \textit{end} for each open node. 
After computation, we obtain the ranges shown in Table~\ref{table:rangesresult}. 

\begin{table}[t]
	\caption{Open node lists with ranges}
	\label{table:rangesresult}
	\centering
	\begin{tabular}{c|cc}
		\hline
		& Left open nodes	& Right open nodes \\
		\hline
		pt$_0$	& []	& [A$_0$(0,4)] \\
		pt$_1$	& [A$_0$(0,4)]	& [A$_0$(0,4), B$_6$(1,3), B$_7$(1,2)] \\
		pt$_2$	& [A$_0$(0,4), B$_6$(1,3), B$_7$(1,2)]	& [A$_0$(0,4), B$_6$(1,3), D$_1$$_3$(2,3)] \\
		pt$_3$	& [A$_0$(0,4), B$_6$(1,3), D$_1$$_3$(2,3)]	& [A$_0$(0,4)] \\
		pt$_4$	& [A$_0$(0,4)]	& [] \\
		\hline
	\end{tabular}
\end{table}

By using these ranges, we can locate the matching nodes of the same node on the different partial trees.
For example, the range of A$_0$ is (0, 4), that means we can 
locate the same nodes of \Nc{A}{0} from pt$_0$ to pt$_4$.
As we can see, there are \Nr{A}{0}, \Np{A}{0}, \Np{A}{0}, 
\Np{A}{0}, and \Nr{A}{0} on pt$_0$ to pt$_4$, respectively.


% LocalWords: fromWord pdf subtree subtrees pre
